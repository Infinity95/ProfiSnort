\chapter{Große Fehlerbehebungen}


\section{Quantum Toolkit Error}

\begin{enumerate}[leftmargin = *, align=parleft, labelsep=3cm]
  \item[Beschreibung]
    Zufälliges Auftreten von Quantum Toolkit (JavaFX) Exceptions.
  \item[Ursache]
    Graphische Updates werden nicht im FX-Application Thread ausgeführt.
  \item[Behebung]
    Die Betroffenen Methoden wurden in Platform.runLater() geschachtelt.
\end{enumerate}

\section{Fehler im Rendering}

\begin{enumerate}[leftmargin = *, align=parleft, labelsep=3cm]
  \item[Beschreibung]
    Im Rendering waren einige Fehler vorhanden. Diese beinhalteten unter anderem falsch dargestellte Pfeilspitzen, schlechte Performance und fehlerhafte Animationen.
  \item[Ursache]
    Der Jung Renderingprozess ist nicht für Echtzeitrendering, bzw. dynamische Größenveränderung der Knoten ausgelegt.
  \item[Behebung]
    Durch Migrierung des Renderingprozesses von Swing nach JavaFX wurde das Problem behoben.
\end{enumerate}

\section{Snort Absturz bei IPC Verbindungsabbruch}

\begin{enumerate}[leftmargin = *, align=parleft, labelsep=3cm]
  \item[Beschreibung]
    Das Unterbrechen der IPC Verbindung nach einem vorherigen Aufbau führt zum Absturz des Snort Plugins (broken pipe Fehler).
  \item[Ursache]
    Der Absturz wird durch fehlerhaftes Zurücksetzen der Verbindung bei Trennung durch \programname hervorgerufen.
  \item[Behebung]
    Durch Senden einer Disconnectanfrage und Abfangen des broken pipe Fehlers wurde das Problem behoben.
\end{enumerate}