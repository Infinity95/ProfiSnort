\chapter{Testszenarien}

In diesem Kapitel werden die im Pflichtenheft definierten globalen Testszenarien ausgewertet. Aufgrund der Unterschiede vom Pflichtenheft zu den endgültigen Implementierungsergebnissen werden die einzelnen Szenarien angepasst, ihre Abdeckung bezüglich umgesetzer funktionaler Anforderungen bleibt jedoch gleich.

\section{Nicht umsetzbare Funktionstests}

Einige der gewünschten Funktionstests stellten sich als nicht durchführbar heraus, da sie nicht implementiert werden konnten.

\begin{itemize}
  \item (1)  Alternativer Start bzw. (0) Programm Start. Die Wahl Snort/Präprozessor einzuschalten entfällt. Wird zusammengelegt.
  \item (2)  Kommunikationsteilnehmer hinzufügen. Wird in dem (4) Normale Netzwerküberwachung automatisch mehrfach ausgeführt.
  \item (7)  Rückverfolgung. War optional, wurde nicht umgesetzt.
  \item (8)  Timeoutbenachrichtigung. War optional, nicht in der GUI umgesetzt.
  \item (11) Netzteilnehmer wird inaktiv. Keine Unterscheidung von aktiven zu inaktiven Knoten umgesetzt.
  \item (12) Fehlerhaften Paket. Nicht implementiert.
  \item (13) Blacklist. Wird manuell umgesetzt durch Filter.
  \item (14) Inaktiver geblacklisteter Teilnehmer. Aus Mangel an 11.
\end{itemize}

Alle daraus resultierenden Testszenarien wurden verändert oder in andere integriert. Die FA-Abdeckung bleibt gleich.

\section{Testszenario 1}

\begin{enumerate}[leftmargin = *, align=parleft, labelsep=3cm]
  \item[Beschreibung] Programm starten und beenden.
  \item[Ergebnis] Reibungslos abgelaufen.
  \item[Coverage] 24\% Zeilen, 39\% Klassen.
\end{enumerate}

\section{Testszenario 2}

\begin{enumerate}[leftmargin = *, align=parleft, labelsep=3cm]
  \item[Beschreibung] Beispielnetzwerk mit 10 Knoten zufällig kommunizieren lassen und Detailfenster zu einem Knoten aufrufen.
  \item[Ergebnis] Reibungslos abgelaufen. Informationsdarstellung korrekt.
  \item[Coverage] 41\% Zeilen, 54\% Klassen.
\end{enumerate}

\section{Testszenario 3}

\begin{enumerate}[leftmargin = *, align=parleft, labelsep=3cm]
  \item[Beschreibung] Beispielnetzwerk mit 10 Knoten, Paketlogs eines Knoten anezeigen lassen.
  \item[Ergebnis] Reibungslos abgelaufen. Es bestehen 2 Möglichkeiten die Logs zu lesen, jeweils korrekt.
  \item[Coverage] 48\% Zeilen, 54\% Klassen.
\end{enumerate}

\section{Testszenario 4}

\begin{enumerate}[leftmargin = *, align=parleft, labelsep=3cm]
  \item[Beschreibung] Filter auf Netzwerk anwenden. Eine MAC-Adresse wird gefiltert.
  \item[Ergebnis] Reibungslos abgelaufen.
  \item[Coverage] 50\% Zeilen, 56\% Klassen.
\end{enumerate}

\section{Testszenario 5}

\begin{enumerate}[leftmargin = *, align=parleft, labelsep=3cm]
  \item[Beschreibung] Graphische Änderung des Netzwerks in der GUI, Verschieben von Knoten, Formalgorithmus auf Graph anwenden per Hotkey, Zoomfunktion. Detailfenster überprüfen.
  \item[Ergebnis] Merkliche Performanceeinbuße beim Algorithmus. Funktionalität dennoch vollständig und richtig.
  \item[Coverage] 44\% Zeilen, 54\% Klassen.
\end{enumerate}

\section{Testszenario 6}

\begin{enumerate}[leftmargin = *, align=parleft, labelsep=3cm]
  \item[Beschreibung] Programmabsturz durch Prozesstermination.
  \item[Ergebnis] Bereits erstellte Filter bleiben vorhanden sind aber deaktiviert in der View. Programm beendet mit Exit Code 0. Snort und spp\_profinet nicht beeinflusst. Test als Erfolg einzustufen.
  \item[Coverage] 54\% Zeilen, 54\% Klassen.
\end{enumerate} 

\section{Testszenario 7}

\begin{enumerate}
  \item[Beschreibung] Zufälliges Klicken in der GUI und kombinieren von Aktionen in zufälligem Muster.
  \item[Ergebnis] Keine Auffälligkeiten. Programm funktioniert weiterhin einwandfrei.
  \item[Coverage] 61\% Zeilen, 56\% Klassen.
\end{enumerate}