\chapter{Globale Testfälle}

\section{Funktionstests zu prüfen}

\begin{description}[style=multiline, leftmargin=4cm, labelwidth=4cm]
  \item[\namedlabel{start}{Programm starten}] Der Nutzer startet \gls{programname}, wird gefragt ob er Snort mit dem \gls{praeprozessor} neu starten will, da dieser noch nicht läuft, stellt im Einstellungsmenü die Länge des Kommunikationslogs auf 10 Minuten.
  \item[\namedlabel{addNetNode}{Kommunikationsteilnehmer hinzufügen}] Ein Kommunikationsteilnehmer wird zum Netzwerk hinzugefügt indem er anfängt \gls{profinet}-\glspl{paket} zu senden. Er erscheint auf der \gls{gui}, Gerätename/\gls{ip}-Adresse/\gls{mac}-Adresse ist direkt sichtbar. Der Nutzer lässt sich mit einem Klick die Knoten-Details anzeigen.
  \item[\namedlabel{normalWatch}{Normale Netzwerküberwachung}] Es kommuniziert ein Beispielnetzwerk von 5 Teilnehmern über \gls{profinet}, in der \gls{gui} baut sich ein Graph selbständig auf und zeigt jeden Kommunikationsteilnehmer sowie die zugehörigen Verbindungen an. Alle Kommunikationswege werden als legal gekennzeichnet. Der Nutzer lässt sich die \gls{log}-Dateien der letzten 10 Minuten anzeigen.
  \item[\namedlabel{guiDisplay}{Korrekte GUI Darstellung}] Ein Paketsender sendet den gleichen Datenverkehr zu einem zweiten, neuen Empfänger, die \gls{gui} zeigt dies an. Weiter sind die \glspl{log} der Verbindungen, welche in der \gls{gui} durch den Nutzer abgerufen werden, bis auf die Adressen identisch zwischen beiden Empfängern.
  \item[\namedlabel{filter}{Filter anwenden}] Der Nutzer stellt im Filtereinstellungsmenü den Namensfilter ein und lässt sich alle Knoten mit Unternamen “Haribo” markieren. \\Die \gls{gui} graut alle anderen Knoten aus.
  \item[\namedlabel{guiChanging}{Graph verändern}] Der Nutzer skaliert den Graphen klein(zoomt raus). \\Der Nutzer wählt einen alternativen Graph-Darstellungsalgorithmus.
  \item[\namedlabel{inactive}{Netzteilnehmer wird inaktiv}] Ein Paketsender wird aus dem Netz entfernt (hört auf zu senden), er wird nach zuvor vom Nutzer eingestellten 5 Sekunden als inaktiv markiert.
  \item[\namedlabel{errpak}{Fehlerhaftes Paket}] Ein fehlerhaftes \gls{paket} wird von einem Paketsender gesendet, der Nutzer lässt sich die \gls{log}-Dateien anzeigen und sieht das als fehlerhaft markierte \gls{paket}.
  \item[\namedlabel{blacklist}{Blacklist}] Ein Angreifer mit \gls{blacklist}-Adresse im Netzwerks löst Illegale-Kommunikation-Alarm aus und erscheint auf der \gls{gui} inklusive Warnung für den Nutzer. Die Kommunikation wird als illegal angezeigt.
  \item[\namedlabel{inactiveBlacklist}{Inaktiver geblacklisteter Teilnehmer}] Nach \gls{blacklist} hört der Angreifer auf zu kommunizieren. Daraufhin wird er als inaktiv aber weiterhin als Angreifer angezeigt. Seine \glspl{log} werden separat gespeichert und sind für den Nutzer lesbar.
  \item[\namedlabel{Exit}{Exit}] \gls{programname} (Programm) beenden.
  \item[\namedlabel{Absturz}{Absturz}] \gls{programname}-Prozess wird terminiert. Es muss sichergestellt sein, dass keine Dateninkonsistenzen auftreten.
\end{description}

\section{Testszenarien}

\begin{itemize}
  \item \ref{start}, \ref{Exit}
  \item \ref{start}, \ref{addNetNode}, \ref{Exit}
  \item \ref{start}, \ref{normalWatch}, \ref{Exit}
  \item \ref{start}, \ref{normalWatch}, \ref{guiDisplay}, \ref{Exit}
  \item \ref{start}, \ref{filter}, \ref{Exit}
  \item \ref{start}, \ref{normalWatch}, \ref{guiChanging}, \ref{Exit}
  \item \ref{start}, \ref{normalWatch}, \ref{inactive}, \ref{Exit}
  \item \ref{start}, \ref{addNetNode}, \ref{errpak}, \ref{Exit}
  \item \ref{start}, \ref{normalWatch}, \ref{blacklist}, \ref{inactiveBlacklist}, \ref{Exit}
  \item \ref{start}, \ref{normalWatch}, \ref{Absturz}
  \item \ref{start}, \ref{normalWatch}, \ref{filter}, \ref{Absturz}, \ref{start}, \ref{addNetNode}, \ref{filter}, \ref{errpak}, \ref{Absturz}, \ref{start}, \ref{normalWatch}, \ref{Exit}
\end{itemize}
