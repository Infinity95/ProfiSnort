\chapter{Zielbestimmung}

Profisnort und PSAnalyzer sollen eine Möglichkeit schaffen, das Intrusion-Detection-System Snort auch mit ProfiNET verwenden zu können.
Hierbei soll PSAnalyzer eine übersichtliche Darstellung der Beziehungen und Datenflüsse in dem überwachten Netzwerk bieten.

\section{Musskriterien}

\subsection{Profisnort}
\begin{itemize}
  \item Das Plugin muss Ethernet-ProfiNet-Pakete erkennen und dekodieren können.
  \item Das Plugin muss ein Ausgabeinterface bereitstellen damit PSAnalyzer die dekodierten Paketdaten empfangen und weiterverarbeiten kann.
\end{itemize}

\subsection{PSAnalyzer}
\begin{itemize}
  \item Das Analyseprogramm muss eine grafische Oberfläche mit folgenden Funktionen bieten:
  \begin{itemize}
    \item Übersichtliche, dynamische Graph-Darstellung der einzelnen Komponenten eines Netzwerks.
    \item Möglichkeit zur zusätzlichen Darstellung von detaillierten Informationen.
  \end{itemize}
  \item Das Analyseprogramm muss die Möglichkeit zur Datenhaltung für möglich spätere Auswertung bieten. 
\end{itemize}

\section{Wunschkriterien}
\subsection{Profisnort}
\begin{itemize}
  \item Das Plugin kann decodierte Pakete Snort zum prüfen mit gewissen Regeln zur Verfügung stellen. 
\end{itemize}
\subsection{PSAnalyzer}
\begin{itemize}
  \item Das Analysetool kann dem Nutzer ermöglichen, den Graphen per Drag-and-Drop zu beeinflussen. 
  \item Das Analysetool kann dem Nutzer grafische Statistiken des bisher stattgefundenen Netzwerkverkehrs anzeigen. 
  \item Das Analysetool kann von Snort ausgelöste Alerts im Graph darstellen. 
\end{itemize}