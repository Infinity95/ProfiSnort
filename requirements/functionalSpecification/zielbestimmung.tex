\chapter{Zielbestimmung}

\gls{programname} und \gls{sppname} sollen eine Möglichkeit schaffen, das \gls{ids} \gls{snort} auch mit \gls{profinet} verwenden zu können.
Hierbei soll \gls{programname} eine übersichtliche Darstellung der Beziehungen und Datenflüsse in dem überwachten Netzwerk bieten.

\section{Musskriterien}

\subsection{\gls{sppname} (Snort-Präprozessor)}

\begin{itemize}
\item Der \gls{praeprozessor} muss \gls{ethernet} \gls{profinet} Pakete erkennen und dekodieren können.

\item Der \gls{praeprozessor} muss ein Ausgabeinterface bereitstellen damit \gls{programname} die dekodierten Paketdaten empfangen und weiterverarbeiten kann.

\item Der \gls{praeprozessor} muss die \gls{profinet} \gls{headerdaten} zusammenfassen und an \gls{programname} weitergeben.
\end{itemize}

\subsection{\gls{programname} (Analyseprogramm)}

\begin{itemize}
\item Das Analyseprogramm muss eine grafische Oberfläche mit folgenden Funktionen bieten:

    \subitem Übersichtliche, dynamische Graphdarstellung der einzelnen Komponenten eines Netzwerks.

    \subitem Möglichkeit zur zusätzlichen Darstellung von detaillierten Informationen.

    \subitem Möglichkeit, den Graphen zu skalieren.
    
    \subitem Es gibt ein Einstellungsmenü, welches dem Nutzer ermöglicht grundlegende Einstellungen vorzunehmen.

\item Legale und illegale Kommunikation müssen unterschiedlich dargestellt werden.

\item Es gibt eine Möglichkeit Bestimmte Subnetze auf eine \gls{blacklist} bzw. \gls{whitelist} zu setzen.

\item Über ein Einstellungsmenü sollen Darstellungsvarianten spezifiziert werden können.

\item Das Analyseprogramm muss die Möglichkeit zur Datenhaltung für mögliche spätere Auswertung bieten.

\item Das Programm funktioniert eigenständig und hat möglichst wenige Abhängigkeiten.

\item Der Paketdurchsatz kann in einem Detailfenster in textueller Form eingesehen werden.

\item Der Übertragungstyp kann, soweit bekannt, im Graph dargestellt werden.
\end{itemize}

\section{Kannkritierien}

\subsection{\gls{sppname}}

\begin{itemize}

\item Der \gls{praeprozessor} kann dekodierte Pakete \gls{snort} zur Verfügung stellen, damit sie dort weiterverarbeitet werden können.

\item Der \gls{praeprozessor} kann \gls{profinet} \gls{tcp} Pakete erkennen und dekodieren.

\item Der \gls{praeprozessor} ist in der Lage andere Protokollarten zur Weiterverarbeitung an \gls{programname} weiterzugeben.
\end{itemize}

\subsection{\gls{programname}}

\begin{itemize}
\item Das Analysetool kann dem Nutzer grafische Statistiken des bisher stattgefundenen Netzwerkverkehrs anzeigen.

\item Das Analysetool kann von \gls{snort} ausgelöste \gls{alert} im Graph darstellen.

\item Die Kommunikationswege und -richtungen können im Graphen dargestellt werden.

\item Der Paketdurchsatz kann im Graph grafisch dargestellt werden (z. B. Farbe der Kante)

\item Ausgabe eines \gls{log} mit Informationen zur Kommunikation bei Klicken auf eine Kante zwischen zwei Netzkomponenten.

\item Dem Benutzer wird ermöglicht, Knoten anhand von Kriterien zu filtern.

\item Die Maximalgröße des aufzuzeichnenden \glspl{log} kann in dem Einstellungsmenü festgelegt werden.

\item Der Benutzer hat die Möglichkeit über ein Menü verschiedene Algorithmen zur Graphdarstellung auszuwählen.
\end{itemize}

\section{Abgrenzungskriterien}

\subsection{\gls{sppname}}
\begin{itemize}
\item Der \gls{praeprozessor} soll nicht die Arbeit von \gls{snort} und dem Analysetool übernehmen.

\item Der \gls{praeprozessor} soll nur dekodieren und nicht analysieren.

\end{itemize}

\subsection{\gls{programname}}
\begin{itemize}

\item \gls{programname} bietet keinen aktiven Schutz vor Angriffen, sondern zeigt solche nur an. (dient lediglich der Analyse und Visualisierung des Netzwerkverkehrs)

\item Die Graphalgorithmen  werden einer Bibliothek entnommen und nicht selbst entwickelt.

\item Die \gls{gui} wird nicht von Grund auf neu entwickelt, sondern verwendet Bibliotheken um die Entwicklung zu erleichtern.

\end{itemize}
