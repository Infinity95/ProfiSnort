\chapter{Zielbestimmung}

Profisnort und PSAnalyzer sollen eine Möglichkeit schaffen, das Intrusion-Detection-System Snort auch mit ProfiNET verwenden zu können.
Hierbei soll PSAnalyzer eine übersichtliche Darstellung der Beziehungen und Datenflüsse in dem überwachten Netzwerk bieten.

\section{Musskriterien}

\subsection{Profisnort}
\begin{itemize}
  \item Das Plugin muss Profinet-Pakete erkennen und dekodieren können.
  \item Das Plugin muss ein Ausgabeinterface bereitstellen damit PSAnalyzer die dekodierten Paketdaten empfangen und weiterverarbeiten kann.
  \item Snort soll in der Lage sein die dekodierten Pakete weiter zu verarbeiten und gegebenenfalls Regeln darauf anwenden.
  \item Falls Snort einen Alarm auslöst, soll dieser zusätzlich an das Analysetool weitergegeben werden.
\end{itemize}

\subsection{PSAnalyzer}
\begin{itemize}
  \item Das Analyseprogramm soll eine grafische Oberfläche mit folgenden Funktionen bieten:
  \begin{itemize}
    \item Übersichtliche Graph-Darstellung der einzelnen Komponenten eines Netzwerks.
    \item Verwalten der Komponenten über ein Rechts-Klick-Menü (entfernen, zurücksetzen, etc.)
    \item Neuanordnung der elemente nach typischen Graph-Layouts.
  \end{itemize}
  \item Das Programm soll in der Lage sein, Protokolle zu speichern und zu einem späteren Zeitpunkt noch einmal abzuspielen.
\end{itemize}

\section{Wunschkriterien} 