\chapter{Zielbestimmung}

\programname und PSAnalyzer sollen eine Möglichkeit schaffen, das Intrusion-Detection-System Snort auch mit PROFINET verwenden zu können.
Hierbei soll PSAnalyzer eine übersichtliche Darstellung der Beziehungen und Datenflüsse in dem überwachten Netzwerk bieten.

\section{Musskriterien}

\subsection{\sppname (Snort Präprozessor)}

\begin{itemize}
\item Der Präprozessor muss Ethernet-PROFINET-Pakete erkennen und dekodieren können.

\item Der Präprozessor muss ein Ausgabeinterface bereitstellen damit \programname die dekodierten Paketdaten empfangen und weiterverarbeiten kann.

\item Der Präprozessor muss die PROFINET Header Daten zusammenfassen und an \programname weitergeben.
\end{itemize}

\subsection{\programname (Analyseprogramm)}

\begin{itemize}
\item Das Analyseprogramm muss eine grafische Oberfläche mit folgenden Funktionen bieten:

\item Übersichtliche, dynamische Graph-Darstellung der einzelnen Komponenten eines Netzwerks.

\item Möglichkeit zur zusätzlichen Darstellung von detaillierten Informationen.

\item Legale und illegale Kommunikation müssen unterschiedlich dargestellt werden.

\item Es gibt eine Möglichkeit Bestimmte Subnetze auf eine Blacklist bzw. Whitelist zu setzen.

\item Möglichkeit, den Graphen zu skalieren.

\item Über ein Einstellungsmenü sollen Darstellungsvarianten spezifiziert werden können.

\item Das Analyseprogramm muss die Möglichkeit zur Datenhaltung für mögliche spätere Auswertung bieten.
\end{itemize}

\section{Kannkritierien}

\subsection{\sppname}

\begin{itemize}

\item Der Präprozessor kann decodierte Pakete Snort zur Verfügung stellen, damit sie dort weiterverarbeitet werden können.

\item Der Präprozessor kann PROFINET TCP Pakete erkennen und dekodieren.

\item Der Präprozessor ist in der Lage andere Protokollarten zur Weiterverarbeitung an \programname weiterzugeben.
\end{itemize}

\subsection{\programname}

\begin{itemize}
\item Es gibt ein Einstellungsmenü welches dem Nutzer ermöglicht grundlegende Einstellungen vorzunehmen.

\item Das Analysetool kann dem Nutzer grafische Statistiken des bisher stattgefundenen Netzwerkverkehrs anzeigen.

\item Das Analysetool kann von Snort ausgelöste Alerts im Graph darstellen.

\item Die Kommunikationswege und -richtungen können im Graphen dargestellt werden.

\item Der Paketdurchsatz kann im Graph dargestellt werden (z. B. Farbe der Kante)

\item Der Übertragungstyp kann, soweit bekannt, im Graph dargestellt werden.

\item Ausgabe eines Kommunikationslogs bei Klicken auf eine Kante zwischen zwei Netzkomponenten.

\item Dem Benutzer wird ermöglicht, Knoten anhand von Kriterien zu filtern.

\item Die Maximalgröße des aufzuzeichnenden Logs kann in dem Einstellungsmenü festgelegt werden.

\item Der Benutzer hat die Möglichkeit über ein Menü verschiedene Algorithmen zur Graphdarstellung auszuwählen.

\item Das Programm funktioniert eigenständig und hat möglichst wenige Abhängigkeiten.
\end{itemize}

\section{Abgrenzungskriterien}

\subsection{\sppname}
\begin{itemize}
\item Das Plugin soll nicht die Arbeit von Snort und dem Analysetool übernehmen.

\item Das Plugin soll nur dekodieren und nicht analysieren.

\end{itemize}

\subsection{\programname}
\begin{itemize}

\item \programname bietet keinen aktiven Schutz vor Angriffen, sondern zeigt solche nur an. (dient lediglich der Analyse und Visualisierung des Netzwerkverkehrs)

\item Die Graphalgorithmen  werden einer Bibliothek entnommen und nicht selbst entwickelt.

\item Die GUI wird nicht von Grund auf neu entwickelt, sondern verwendet Bibliotheken um die Entwicklung zu erleichtern.

\end{itemize} 