\chapter{Nichtfunktionale Anforderungen}

\section{\gls{sppname}}
\begin{tabular}{lp{0.9\linewidth}}
NFA10: & Es darf kein Paketrückstau durch zu langsame Bearbeitung der \glspl{paket} entstehen. \\

NFA20: & Der \gls{praeprozessor} darf keine Abstürze von \gls{snort} verursachen. \\

NFA30: & \glspl{paket} werden innerhalb von 1ms verarbeitet.
\end{tabular}
\section{\gls{programname}}
\begin{tabular}{lp{0.9\linewidth}}
NFA40: & Die \gls{gui} zeigt ein Gerät spätestens eine Sekunde nach Paketregistrierung im Netzwerk an. \\

NFA50: & Ein einkommendes \gls{paket} ist nach höchstens 10ms bearbeitet. \\

NFA60: & Das Programm unterstützt mindestens 100 Teilnehmer. \\

NFA70: & Bei Benutzereingaben muss das Programm innerhalb von 500ms reagieren. \\

NFA80: & \gls{programname} soll möglichst unabhängig von \gls{sppname} agieren. \\

NFA90: & \gls{programname} soll eine möglichst simple Schnittstelle zu \gls{sppname} haben um auch Daten von anderen Netzwerktools darstellen zu können. \\

NFA100: & Das Programm soll so gebaut sein, dass ein einfacher Sprachwechsel der \gls{gui} möglich ist. \\
\end{tabular}
