\chapter{Nichtfunktionale Anforderungen}

\section{\sppname}
\begin{tabular}{lp{0.9\linewidth}}
NFA10: & Es darf kein Paketrückstau durch zu langsame Bearbeitung der Pakete entstehen. \\

NFA20: & Der Präprozessor darf keine Abstürze von Snort verursachen. \\

NFA30: & Pakete werden innerhalb von 1ms verarbeitet.
\end{tabular}
\section{\programname}
\begin{tabular}{lp{0.9\linewidth}}
NFA40: & Die GUI zeigt ein Gerät spätestens eine Sekunde nach Paketregistrierung im Netzwerk an. \\

NFA50: & Ein einkommendes Paket ist nach höchstens 10ms bearbeitet. \\

NFA60: & Das Programm unterstützt mindestens 100 Teilnehmer. \\

NFA70: & Bei Benutzereingaben muss das Programm innerhalb von 500ms reagieren. \\

NFA80: & \programname soll möglichst unabhängig von \sppname agieren. \\

NFA90: & \programname soll eine möglichst simple Schnittstelle zu \sppname haben um auch Daten von anderen Netzwerktools darstellen zu können. \\
\end{tabular}
