\chapter{Nichtfunktionale Anforderungen}

\section{\gls{sppname}}

\newcommand\nfa[2]{\namedlabel{nfa#1}{NFA#1}: & #2 \\}


\begin{tabular}{lp{0.9\linewidth}}

\nfa{10}{Es darf kein Paketrückstau durch zu langsame Bearbeitung der \glspl{paket} entstehen.}

\nfa{20}{Der \gls{praeprozessor} darf keine Abstürze von \gls{snort} verursachen.}

\nfa{30}{\glspl{paket} werden innerhalb von 1ms verarbeitet.}

\end{tabular}

\section{\gls{programname}}

\begin{tabular}{lp{0.9\linewidth}}
\nfa{40}{Die \gls{gui} zeigt ein Gerät spätestens eine Sekunde nach Paketregistrierung im Netzwerk an.}

\nfa{50}{Ein ankommendes \gls{paket} ist nach höchstens 10ms bearbeitet.}

\nfa{60}{Das Programm unterstützt mindestens 100 Teilnehmer.}

\nfa{70}{Bei Benutzereingaben muss das Programm innerhalb von 500ms reagieren.}

\nfa{80}{\gls{programname} soll möglichst unabhängig von \gls{sppname} agieren.}

\nfa{90}{\gls{programname} soll eine möglichst simple Schnittstelle zu \gls{sppname} haben um auch Daten von anderen Netzwerktools darstellen zu können.}

\nfa{100}{Das Programm soll so gebaut sein, dass ein einfacher Sprachwechsel der \gls{gui} möglich ist.}
\end{tabular}
