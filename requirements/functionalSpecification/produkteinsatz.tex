\chapter{Produkteinsatz}

\section{Anwendungsbereiche}
\gls{programname} soll zur Visualisierung von industriellen Netzwerken und zur potentiellen Gefahrenerkennung eingesetzt werden.

\section{Zielgruppen}
Institute und Forschungsgruppen, die sich mit Sicherheit in der Industrie und der Visualisierung von komplexen Netzen beschäftigen. Das Programm richtet sich auch an Menschen mit Grundverständnis der IT-Sicherheit, die sich mit dem Thema vertraut machen möchten.\newline\newline
Industrielle Anlagen, die eine laufende Übersicht ihres Datenverkehrs sehen möchten. Man könnte sich dabei vorstellen, dass \gls{programname} dabei die Funktion einer Überwachungskamera übernimmt, da sich die Netzwerkaktivität in Echtzeit überwachen lässt und zusätzlich die Möglichkeit besteht Aktivitäten aus der Vergangenheit anhand aufgezeichneter \glspl{log} zurück zu verfolgen.

\section{Betriebsbedingungen}
Das Programm ist für eine industrielle Betriebsumgebung mit \gls{profinet} als Kommunikationsprotokoll ausgelegt. Es ist hauptsächlich für \gls{linux} Betriebssysteme ausgelegt und läuft nur auf diesen garantiert.
