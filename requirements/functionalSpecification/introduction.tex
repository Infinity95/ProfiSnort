\chapter*{Einleitung}

In der Industrie gewinnt Vernetzbarkeit immer mehr Bedeutung. Auch das Vernetzen von externen Geräten mit den Produktions-Maschinen wird immer gebräuchlicher, um die Arbeitseffizienz zu erhöhen. Um diese Kommunikation zu vereinfachen und verbessern wurde das Process Field Network (im Folgenden PROFINET) entwickelt. PROFINET setzt auf Ethernet für echtzeitfähige Anwendungen und TCP/IP für langsamere IO Anwendungen. Durch die immer mehr vernetzten und oft dem Internet zugänglichen Produktionsstätten ist Sicherheit mittlerweile von höchster Bedeutung. Dieses Projekt setzt an dieser Stelle an und soll dem Intrusion Detection System Snort ermöglichen, die PROFINET Protokolle zu verstehen und zu verarbeiten. Um dem Benutzer eine Übersicht über die stattfindenden Kommunikationsprozesse zu verschaffen, soll eine GUI entwickelt werden. Diese soll mithilfe verschiedener Graphalgorithmen eine geordnete Darstellung ermöglichen. 