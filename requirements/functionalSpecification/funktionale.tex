\chapter{Funktionale Anforderungen}

\renewcommand{\arraystretch}{2}
\section{\gls{sppname}}
\begin{tabular}{lp{0.9\linewidth}}

FA10: & \textit{Erkennung fehlerhafter Pakete: }Der \gls{praeprozessor} muss fehlerhafte Pakete erkennen und diese als solche markieren können, bevor sie an das Analysetool weitergereicht werden. \\

FA20: & \textit{Erkennen von \gls{profinet} \glspl{paket}n: }Der \gls{praeprozessor} muss \gls{profinet} \glspl{paket} erkennen können. \\

FA30: & \textit{Dekodieren von \gls{profinet} \glspl{paket}n: }Der \gls{praeprozessor} muss die erkannten \glspl{paket} dekodieren können. \\

FA40: & \textit{Unterscheiden der verschiedenen \gls{frameid}s: }Der \gls{praeprozessor} muss die verschiedenen von \gls{profinet} spezifizierten \gls{frameid}s unterscheiden können.\\

FA50: & \textit{Output Kanal zu \gls{programname}: }Es muss eine \gls{interprocess} für die Weitergabe der dekodierten Daten möglich sein.\\

FA70: & \textit{Ausgabeschnittstelle: }\gls{sppname} muss eine einheitliche Schnittstelle zum Übertragen der Daten an \gls{programname} verwenden. Diese Schnittstelle muss prinzipiell auf andere Protokolle erweiterbar sein.\\

\end{tabular}

\section{\gls{programname}}
\subsection{Funktionalität}

\begin{longtable}{lp{0.9\linewidth}}

FA90: & \textit{Starten von \gls{snort}: }Wenn \gls{snort} noch nicht läuft, wird der Nutzer gefragt, ob \gls{snort} gestartet werden soll. \\

FA100: & \textit{\gls{sppname} anschalten: }Wenn \gls{snort} nicht läuft und \gls{sppname} in der \gls{snort} Konfiguration deaktiviert ist, wird der Benutzer gefragt ob, er dieses aktivieren möchte. Falls der Benutzer \gls{sppname} nicht aktiviert, beendet sich \gls{programname}. \\

FA110: & \textit{\gls{snort} läuft aber \gls{sppname} ist deaktiviert: }Falls \gls{snort} schon läuft aber \gls{sppname} deaktiviert ist, hat der Benutzer die Wahl, \gls{sppname} zu aktivieren oder \gls{programname} zu beenden. Falls der \gls{praeprozessor} aktiviert werden soll, wird \gls{snort} mit dem reload Befehl neu geladen. \\

FA120: & \textit{Empfangen von Paketinformationen: }\gls{programname} muss Paketinformationen empfangen können, die dem in FA70 genannten Interface entsprechen. \\

FA130: & \textit{Erkennen neuer Netzwerkteilnehmer: }Das Programm erkennt neue Netzwerkteilnehmer und überprüft ihre Legalität (siehe FA150). \\

FA132: & \textit{Erkennen neuer Netzwerkkommunikation: }Das Programm erkennt aufkommende Kommunikation im betrachteten Netzwerk. \\

FA133: & \textit{Erstellen neuer Kanten: }Wenn eine neue Kommunikation erkannt wurde, wird diese im Graphen durch eine gerichtete Kante repräsentiert, wobei die Richtung der Kante vom Sender zum Empfänger zeigt. \\

FA135: & \textit{Erstellen neuer Knoten: }Wenn ein neuer Netzwerkteilnehmer erkannt wurde, wird dieser im Graphen durch einen neuen Knoten repräsentiert. \\

FA140: & \textit{Erkennen von Gerätenamen: }Das Analysetool soll Gerätenamen, falls vorhanden, erkennen und dem entsprechenden Knoten hinzufügen. \\

FA150: & \textit{Kennzeichnen illegaler Knoten: }Ein Knoten, der ein Gerät aus einem illegalen Adressraum repräsentiert oder eine unbekannte/unerwünschte/illegale Identität (nicht auf der \gls{whitelist}) hat, wird für den Benutzer auffällig gekennzeichnet. \\

FA160: & \textit{Kennzeichnen illegaler Kanten: }Kanten, die eingehende und ausgehende Kommunikation eines illegalen Knotens repräsentieren, werden für den Benutzer auffällig gekennzeichnet. \\

FA170: & \textit{Erkennen der Kommunikationswege: }Das Programm muss basierend auf Adressen/Gerätenamen einen Kommunikationsgraph erstellen können. \\

FA180: & \textit{Paketaustausch Statistik: }Das Programm muss berechnen können, wie viele \glspl{paket} einzelne Knoten pro Sekunde senden bzw. empfangen. Diese Information muss in dem Fenster zur detaillierten Knotenstatistik darstellbar sein. \\

FA190: & \textit{Gesamtzahl der Pakete: }Das Programm muss die gesamte Zahl der empfangenen und gesendeten \glspl{paket} in dem Fenster zur detaillierten Knotenstatistik darstellen können. \\

FA220: & \textit{Zeichnen des Graphen: }Der Graph kann mittels Graphalgorithmen (z.B. Kamada-Kawai, Fruchterman-Reingold) dargestellt werden. \\

FA250: & \textit{Kantenunterschied: }Die unterschiedlichen \gls{profinet} Protokolle werden durch Labels oder Farben voneinander unterscheidbar sein. \\

FA260: & \textit{Logging: }Das Programm erstellt ein \gls{log} aller \glspl{paket} mit kritischen Informationen und kann dieses auf Benutzerwunsch als Textdatei ausgeben. \\

FA270: & \textit{Spezifisches Logging: }Das Programm erstellt spezifische \glspl{log} zu einzelnen Knoten mit den von diesem Knoten gesendeten/empfangenen \glspl{paket}n. \\

FA280: & \textit{Fehlerhafte Pakete: }Von \gls{sppname} als fehlerhaft erkannte \glspl{paket} werden mit einem \gls{fehlerflag} versehen und so in das \gls{log} übernommen.

\end{longtable}

\subsection{Optionale Funktionalität}
\begin{longtable}{lp{0.9\linewidth}}

FA290: & \textit{Rückverfolgung: }Der Benutzer kann den zeitlichen Ablauf grafisch zurückverfolgen (wie ein Video) und sämtliche Kommunikationswege erneut abspielen. \\

FA300: & \textit{Kennzeichnen inaktiver Knoten: }Knoten werden nach einer im Einstellungsmenü spezifizierten Zeit als inaktiv gekennzeichnet, wenn sie in dieser Zeit keine \glspl{paket} empfangen oder gesendet haben. \\

FA305: & \textit{Darstellung des Paketvolumens: }Das Paketvolumen einzelner Verbindungen ist im Graph visuell darstellbar. Diese Darstellung kann in den Einstellungen der Übersicht halber deaktiviert werden. \\

FA310: & \textit{Abgrenzen gefilterter Knoten: }Alle Knoten, die nicht den Filteroptionen genügen, werden optisch vom Rest des Graphen getrennt. \\

FA315: & \textit{Timeoutbenachrichtigung: }Wenn ein \gls{iocontroller} einen \gls{identifyrequest} sendet und ein Timeout stattfindet, wird der Benutzer grafisch darüber in Kenntnis gesetzt.\\

FA317: & \textit{Darstellen von \gls{snort} Alerts: }Falls \gls{snort} Alerts auftreten, werden diese grafisch dargestellt.

\end{longtable}
\pagebreak
\subsection{Datenhaltung}

\begin{longtable}{lp{0.9\linewidth}}

FA320: & \textit{Speichern des Datenverkehrs: }Der Datenverkehr wird in einem Datenbestand gespeichert. \\

FA330: & \textit{Kategorisierung der Daten: }Die Daten werden nach den verschiedenen Kommunikationstypen kategorisiert. \\

FA340: & \textit{Datenausgabe: }Auf Anforderungen des Benutzers müssen Daten ausgelesen und ausgegeben werden. \\

FA350: & \textit{Zeitraum der Datenspeicherung: }Der Zeitraum, über den die Daten gespeichert werden sollen, kann in den Einstellungen festgelegt werden. \\

\end{longtable}

\subsection{Benutzerinteraktion}

\begin{longtable}{lp{0.9\linewidth}}

FA360: & \textit{Speichern der Einstellungen: }Das Programm speichert nach Änderungen des Benutzers automatisch bzw. nach Klick auf den Speichern-Knopf die neuen Einstellungen. \\

FA365: & \textit{\gls{gui}: }Es gibt eine Benutzeroberfläche, über welche der Nutzer die Möglichkeit hat mit dem Graphen und den Einstellungen zu interagieren. \\

FA370: & \textit{Fehlerhafte Benutzereingaben: }Die Benutzereingaben im Einstellungsmenü und Filtermenü werden auf Fehler überprüft. Bei Fehlern wird der Benutzer zur neuen Eingabe aufgefordert und der vorherige Wert nicht verändert. \\

FA375: & \textit{Einstellungsmenü: }Es gibt ein Einstellungsmenü welches durch den Benutzer über die \gls{gui} erreichbar ist. \\

FA380: & \textit{Skalierung des Graphen: }Der Benutzer ist in der Lage über das Scrollrad die Skalierung des Graphen zu verändern. \\

FA390: & \textit{Detaillierte Information zu Knoten: }Ein Overlay mit detaillierten Informationen wird bei Klick auf einen Knoten dargestellt. \\

FA400: & \textit{Generelle Informationen: }In einem Overlay werden generelle Informationen über den Graph angezeigt. \\

FA410: & \textit{Hover über einem Knoten: }Hovern erfüllt im Graph dieselben Funktionen wie klicken. Zusätzlich zum aktuell angeklickten Knoten werden zum Vergleich Informationen des aktuell gehoverten Knotens angezeigt. \\

FA420: & \textit{\gls{blacklist}: }Der Benutzer hat die Möglichkeit, bestimmte \gls{mac}- und \gls{ip} Adressräume sowie Gerätenamen als illegal zu markieren. \\

FA425: & \textit{\gls{whitelist}: }Der Benutzer hat die Möglichkeit, bestimmte \gls{mac}- und \gls{ip} Adressräume sowie Gerätenamen als legal zu markieren. \\

FA427: & \textit{Whitelist/Blacklist aktivieren: }Der Benutzer kann über das Einstellungsmenü wählen, welche Listen verwendet werden sollen. \\

\end{longtable}

\subsection{Optionale Benutzerinteraktion}

\begin{longtable}{lp{0.9\linewidth}}

FA430: & \textit{Statistiken zu Knoten: }Ein Overlay mit zeitlichen Statistiken wird bei Klick auf einen Knoten angezeigt. \\

FA440: & \textit{Filtereinstellungen: }Die Filtereinstellungen können im Einstellungsmenü angepasst werden. \\

FA450: & \textit{Filteroptionen: }Die Knoten können nach \gls{ip}, \gls{mac} Adresse, Netzwerkaktivität oder Gerätenamen gefiltert werden. \\

FA460: & \textit{Filter deaktivieren/reaktivieren: }Es gibt einen Hotkey und eine Option im Filterfenster zum aktivieren oder deaktivieren von Filtern. \\

FA470: & \textit{\gls{io-supervisor} festlegen: }Der Benutzer kann Geräte als \gls{io-supervisor} markieren. \\

FA480: & \textit{Verbindungsrichtung: }Wenn mindestens x\% (standardmäßig 90\%, konfigurierbar) der \glspl{paket} aus einer Richtung kommen, werden Verbindungen im Graph gerichtet angezeigt. \\

FA490: & \textit{Graphdarstellungsalgorithmen: }Die Graphdarstellungsalgorithmen können über das Einstellungsmenü ausgewählt/geändert werden. \\

FA500: & \textit{Kategorisierung: }Einzelne Knoten können im Graph als unwichtig eingestuft werden, um diese visuell abzugrenzen. \\
\end{longtable}
