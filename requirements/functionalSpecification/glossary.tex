\newacronym
[description={Die grafische Benutzeroberfläche}]
{gui}{GUI}{Graphical User Interface}

\newacronym
[description={Ein Netzwerkprotokoll, das definiert, auf welche Art und Weise Daten zwischen Computern ausgetauscht werden sollen.}]
{tcp}{TCP}{Transmission Control Protocol}

\newacronym
[description={Input und Output}]
{io}{IO}{Input/Output}

\newacronym
[description={System oder Software zur Entdeckung von Netzwerkangriffen}]
{ids}{IDS}{Intrusion Detection System}

\newacronym
[description={Protokoll zur Übermittlung von Daten im Internet}]
{ip}{IP}{Internet Protocol}

\newacronym
[description={Ist eine vom Institute of Electrical and Electronics Engineers (IEEE) entworfene Erweiterung des OSI-Modells.}]
{mac}{MAC}{Media Access Control}

  \newglossaryentry{alert}{
  name=Alert,
  plural=Alerts,
  description={Ein von Snort versendeter Alarm, welcher durch verschieden Regeln abgefangen und verarbeitet werden kann.}
}

  \newglossaryentry{whitelist}{
  name=Whitelist,
  description={is a programmable machine that receives input,
               stores and manipulates data, and provides
               output in a useful format}
}

  \newglossaryentry{blacklist}{
  name=Blacklist,
  description={is a programmable machine that receives input,
               stores and manipulates data, and provides
               output in a useful format}
}

  \newglossaryentry{ethernet}{
  name=ethernet,
  description={is a programmable machine that receives input,
               stores and manipulates data, and provides
               output in a useful format}
}
  \newglossaryentry{headerdaten}{
  name=Headerdaten,
  description={is a programmable machine that receives input,
               stores and manipulates data, and provides
               output in a useful format}
}
  \newglossaryentry{io-supervisor}{
  name=IO\_Supervisor,
  description={is a programmable machine that receives input,
               stores and manipulates data, and provides
               output in a useful format}
}
  \newglossaryentry{linux}{
  name=Linux,
  description={is a programmable machine that receives input,
               stores and manipulates data, and provides
               output in a useful format}
}
  \newglossaryentry{programname}{
  name=\programname,
  description={is a programmable machine that receives input,
               stores and manipulates data, and provides
               output in a useful format}
}
  \newglossaryentry{netzwerkpakete}{
  name=Netzwerkpaket,
  plural=Netzwerkpakete,
  description={is a programmable machine that receives input,
               stores and manipulates data, and provides
               output in a useful format}
}
  \newglossaryentry{profinet}{
  name=PROFINET,
  description={is a programmable machine that receives input,
               stores and manipulates data, and provides
               output in a useful format}
}
  \newglossaryentry{profinet-tcp}{
  name=PROFINET TCP,
  description={is a programmable machine that receives input,
               stores and manipulates data, and provides
               output in a useful format}
}
  \newglossaryentry{snort}{
  name=Snort,
  description={is a programmable machine that receives input,
               stores and manipulates data, and provides
               output in a useful format}
}
  \newglossaryentry{praeprozessor}{
  name=Präprozessor,
  plural=Präprozessoren,
  description={is a programmable machine that receives input,
               stores and manipulates data, and provides
               output in a useful format}
}
  \newglossaryentry{sppname}{
  name=\sppname,
  description={is a programmable machine that receives input,
               stores and manipulates data, and provides
               output in a useful format}
}
  \newglossaryentry{x86-64)}{
  name=x86\_64,
  description={is a programmable machine that receives input,
               stores and manipulates data, and provides
               output in a useful format}
}



\newglossaryentry{frameid}{
  name=Frame ID,
  description={is a programmable machine that receives input,
               stores and manipulates data, and provides
               output in a useful format}
}

\newglossaryentry{interprocess}{
    name=Interprozesskommunikation,
    description={adfssekjghsalkgtjreah}
}
\newglossaryentry{paket}{
    name=Paket,
    plural=Pakete,
    description={aadsfkdsajgflkahtljewr}
}

\newglossaryentry{log}{
    name=Log,
    plural=Logs,
    description={ageaewhgjreahj}
}

\newglossaryentry{fehlerflag}{
    name=Fehlerflag,
    description={Ein Bit, welches bestimmt ob der Eintrag fehlerhaft ist, oder nicht.}
}

\newglossaryentry{iocontroller}{
    name=IO\_Controller,
    description={}
}

\newglossaryentry{identifyrequest}{
    name=Identify Request,
    description={}
}
