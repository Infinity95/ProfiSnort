\newacronym
[description={Die grafische Benutzeroberfläche}]
{gui}{GUI}{Graphical User Interface}

\newacronym
[description={Ein Netzwerkprotokoll, das definiert, auf welche Art und Weise Daten zwischen Computern ausgetauscht werden}]
{tcp}{TCP}{Transmission Control Protocol}

\newacronym
[description={Input und Output}]
{io}{IO}{Input/Output}

\newacronym
[description={System oder Software zur Entdeckung von Netzwerkangriffen}]
{ids}{IDS}{Intrusion Detection System}

\newacronym
[description={Protokoll zur Übermittlung von Daten im Internet}]
{ip}{IP}{Internet Protocol}

\newacronym
[description={Eine vom Institute of Electrical and Electronics Engineers (IEEE) entworfene Erweiterung des OSI-Modells}]
{mac}{MAC}{Media Access Control}

\newacronym
[description={Methoden zum Informationsaustausch, informatisch gesprochen Datenübertragung, von nebenläufigen Prozessen oder Threads}]
{ipc}{IPC}{Inter-Process Communication}

\newacronym
[description={Ein Muster zur Strukturierung von Software-Entwicklung in die drei Einheiten Datenmodell (engl. model), Präsentation (engl. view) und Programmsteuerung (engl. controller)}]
{mvc}{MVC}{Model View Controler}

\newglossaryentry{alert} {
  name=Alert,
  plural=Alerts,
  description={Ein von Snort versendeter Alarm, welcher durch verschiedene Regeln abgefangen und verarbeitet werden kann}
}

\newglossaryentry{whitelist} {
  name=Whitelist,
  description={Eine Liste auf der sich vertrauenswürdige Geräte befinden. Alle Geräte, die sich nicht auf dieser Liste befinden, werden als potenziell gefährlich gewertet und entsprechend markiert}
}

\newglossaryentry{blacklist}{
  name=Blacklist,
  description={Eine Liste auf der sich \textbf{nicht} vertrauenswürdige Geräte (Knoten) befinden. Alle Geräte, die sich auf dieser Liste befinden, werden markiert, alle anderen nicht}
}

\newglossaryentry{ethernet}{
  name=Ethernet,
  description={Eine Technologie, die Software und Hardware für kabelgebundene Datennetze spezifiziert, welche ursprünglich für lokale Datennetze (LANs) gedacht war und daher auch als LAN-Technik bezeichnet wird}
}

\newglossaryentry{headerdaten}{
  name=Headerdaten,
  description={Die Kopfzeile eines Datenblocks, die Zusatzinformationen beinhaltet (Bsp. IP Addresse in den Headerdaten einer E-mail)}
}

\newglossaryentry{io-supervisor}{
  name=IO\_Supervisor,
  description={Ein \gls{profinet} IO\_Supervisor ist beispielsweise eine Engineering-Station die für Inbetriebnahmezwecke einen temporären Zugriff auf die Feldgeräte besitzen kann}
}

\newglossaryentry{linux}{
  name=Linux,
  description={Freie, unix basierende Betriebssysteme, die auf dem Linux-Kernel basieren. Also eine Betriebssystemart}
}

\newglossaryentry{programname}{
  name=\programname,
  description={Das Netzwerktool worum es sich unter anderem in diesem Dokument handelt. Das Programm dient zur Darstellung des Netzwerkgraphen}
}

\newglossaryentry{profinet}{
  name=PROFINET,
  plural=Netzwerkpakete,
  description={PROFINET (Process Field Network) ist der offene Industrial Ethernet-Standard von Profibus \& Profinet International (PI) für die Automatisierung. Profinet nutzt \gls{tcp}/\gls{ip} und IT-Standards, ist Echtzeit-\gls{ethernet}-fähig und ermöglicht die Integration von Feldbus-Systemen}
}

\newglossaryentry{profinet-tcp}{
  name=PROFINET TCP,
  description={Für NRT (non-real-time) Kommunikation benutzt \gls{profinet} unter anderem das Transmission Control Protocol (TCP)}
}

\newglossaryentry{snort}{
  name=Snort,
  description={Snort ist ein open source \gls{ids} mit realtime Paketanalyse und Paketlogging}
}

\newglossaryentry{praeprozessor}{
  name=Präprozessor,
  plural=Präprozessoren,
  description={Ein Präprozessor ist eine Art Plugin für \gls{snort}. Er wird nach den Decodern und vor der Rules Engine aufgerufen. Er dient zur fortgeschrittenen Paketverarbeitung wie z.B. Normalisierung oder weiteres Decodieren}
}

\newglossaryentry{sppname}{
  name=\sppname,
  description={Der Name des Präprozessors. Der Präprozessor ist dafür zuständig \gls{profinet} \glspl{paket} zu dekodieren}
}

\newglossaryentry{frameid}{
  name=Frame ID,
  description={Ein Realtime-Frame beginnt immer mit einer 2-Byte FrameID. Darin steht in welcher Klasse sich das Frame befindet (z.B. die Geschwindikeitsklasse)}
}

\newglossaryentry{interprocess}{
    name=Interprozesskommunikation,
    description={Kommunikation, die zwischen zwei unabhängigen Prozessen stattfindet}
}

\newglossaryentry{paket}{
    name=Paket,
    plural=Pakete,
    description={Eine in sich geschlossene Dateneinheit, die ein Sender (z.B. ein digitaler Messfühler) oder auch ein sendender Prozess einem Empfänger sendet}
}

\newglossaryentry{log}{
    name=Log,
    plural=Logs,
    description={Eine Logdatei enthält das automatisch geführte Protokoll aller oder bestimmter Aktionen eines Prozesses}
}

\newglossaryentry{fehlerflag}{
    name=Fehlerflag,
    description={Ein Bit, welches bestimmt, ob der Eintrag fehlerhaft ist, oder nicht}
}

\newglossaryentry{iocontroller}{
    name=IO\_Controller,
    description={Ein \gls{profinet} IO-Controller (IO-Controller) hat die Kontrolle über den auf ein oder mehrere Feldgeräte verteilten Prozess. Bei ihm laufen die Prozess-Daten und Alarme ein und werden im Anwenderprogramm verarbeitet. In einer Automatisierungsanlage ist ein IO-Controller normalerweise eine Speicher-Programmierbare-Steuerung (SPS), ein DCS-System oder ein PC. Das Einrichten der Kommunikationswege erfolgt beim System-Hochlauf}
}

\newglossaryentry{identifyrequest}{
    name=Identify Request,
    description={Ein request eines Netzwerkteilnehmers der die Identität eines anderen einfordert}
}

\newglossaryentry{x86}{
  name=x86\_64,
  description={64 Bit Architektur}
}
