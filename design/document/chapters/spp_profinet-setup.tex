\chapter{Aufbau \sppname}

\section{Architekturbeschreibung}

Sinn und Zweck dieses \gls{snort} \gls{praeprozessor}s \sppname ist das Abfangen,
Komprimieren und Weiterleiten speziell von \gls{profinet}-\glspl{paket}n. Diese
sind am \gls{ethertype} erkennbar, er muss der Hexadezimalzahl 0x8892 entsprechen.
Des Weiteren soll der \gls{praeprozessor} über eine Reihe von erweiterbaren
Decodern wichtige Informationen aus dem empfangenen \gls{paket} auslesen und in
eine von uns vordefinierte Datenstruktur names Truffle schreiben. Pro
Netzwerk-\gls{paket} entsteht also ein Truffle. Dieses wird dann per \gls{ipc}
an unser \programname //valentin: ich mach hier weiter//

// An valentin: schreibe noch hin das wir dort sicherheits checks durchfueren,
// sodass wir davon ausgehen koennen, dass unsere truffle struktur sicher ist

\section{Paketbeschreibung}
