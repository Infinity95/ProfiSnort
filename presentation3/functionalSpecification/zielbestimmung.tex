\chapter{Zielbestimmung}

\gls{sppname} soll das \gls{ids} \gls{snort} um das Identifizieren von \gls{profinet} \glspl{paket}n erweitern und dekodierte \glspl{paket} an \gls{programname} senden. Hierbei soll \gls{programname} eine übersichtliche Darstellung der Beziehungen und Datenflüsse des überwachten Netzwerkes bieten.

\section{Musskriterien}

\subsection{\gls{sppname} (Snort-Präprozessor)}

\begin{itemize}
\item Der \gls{praeprozessor} muss \gls{ethernet} \gls{profinet} Pakete erkennen und dekodieren können.

\item Der \gls{praeprozessor} muss ein Ausgabeinterface bereitstellen damit \gls{programname} die dekodierten Paketdaten empfangen und weiterverarbeiten kann.

\item Der \gls{praeprozessor} muss die \gls{profinet} \gls{headerdaten} zusammenfassen und an \gls{programname} weitergeben.
\end{itemize}

\subsection{\gls{programname} (Analyseprogramm)}

\begin{itemize}
\item Das Analyseprogramm muss eine grafische Oberfläche mit folgenden Funktionen bieten:

	\begin{itemize}
    \item Übersichtliche Darstellung der Kommunikationsbeziehungen eines Netzwerks mithilfe eines Graphen.

    \item Netzwerkteilnehmer werden im Graphen als Knoten dargestellt, Kommunikationswege als Kanten.

    \item Möglichkeit zur zusätzlichen Darstellung von detaillierten Informationen (z.B. Paketdurchsatz, letzte Aktivität).

    \item Vergößern und Verkleinern des Graphen (Zoom).

    \item Es gibt ein Einstellungsmenü, welches dem Nutzer ermöglicht grundlegende Einstellungen (bspw. Sprachwahl, Farbwahl) vorzunehmen.
    \end{itemize}

\item Es gibt eine Möglichkeit Bestimmte Subnetze auf eine \gls{blacklist} bzw. \gls{whitelist} zu setzen.

\item Durch White- oder Blacklist als legal bzw. illegal markierte Kommunikation muss unterschiedlich dargestellt werden.

\item \gls{programname} zeichnet Netzwerkkommunikation in \glspl{log} auf.

\item Die Maximalgröße aufzuzeichnender \glspl{log} kann im Einstellungsmenü festgelegt werden.

\item Das Analyseprogramm muss die Möglichkeit zur Datenhaltung und möglichen späteren Auswertung bieten.

\item Das Programm funktioniert eigenständig und hat möglichst wenige Abhängigkeiten.

\item Der Paketdurchsatz kann in einem Detailfenster in textueller Form eingesehen werden.

\item Die Bezeichnung der \gls{frameid} eines Paketes kann, soweit bekannt, im Graph dargestellt werden.

\end{itemize}

\section{Kannkritierien}

\subsection{\gls{sppname}}

\begin{itemize}

\item Der \gls{praeprozessor} kann die dekodierten Pakete \gls{snort} zur Verfügung stellen, damit sie dort weiterverarbeitet werden können.

\item Der \gls{praeprozessor} ist in der Lage andere Protokollarten die \gls{profinet} verwendet zur Weiterverarbeitung an \gls{programname} zu senden (z.B. \gls{tcp}, \gls{udp}, \gls{lldp}).

\item Der \gls{praeprozessor} kann \gls{snort} \glspl{alert} auslösen. Die Rule Engine von \gls{snort} wird dabei nicht verwendet.
\end{itemize}

\subsection{\gls{programname}}

\begin{itemize}
\item Das Analysetool kann dem Nutzer grafische Statistiken des bisher stattgefundenen Netzwerkverkehrs anzeigen.

\item Das Analysetool kann von \gls{snort} ausgelöste \glspl{alert} im Graph darstellen.

\item Die Richtung des Kommunikationsflusses kann im Graphen dargestellt werden.

\item Der Paketdurchsatz kann im Graph visuell gekennzeichnet werden (z.B. Farbe der Kante).

\item Dem Benutzer wird ermöglicht, Knoten anhand von Kriterien zu filtern (z.B. Adresse, Name).

\item Der Benutzer hat die Möglichkeit über ein Menü geeignete Algorithmen zur Darstellung des Graphen auszuwählen.
\end{itemize}

\section{Abgrenzungskriterien}

\subsection{\gls{sppname}}
\begin{itemize}
\item Der \gls{praeprozessor} soll nicht die Arbeit von \gls{snort} und dem Analysetool übernehmen.

\item Der \gls{praeprozessor} soll nur dekodieren und nicht analysieren.

\end{itemize}

\subsection{\gls{programname}}
\begin{itemize}

\item \gls{programname} bietet keinen aktiven Schutz vor Angriffen, sondern zeigt solche nur an (Das Programm dient lediglich der Analyse und Visualisierung des Netzwerkverkehrs).

\item Die Graphalgorithmen  werden einer Bibliothek entnommen und nicht selbst entwickelt.

\item Die \gls{gui} wird nicht von Grund auf neu entwickelt, sondern verwendet Bibliotheken um die Entwicklung zu erleichtern.

\end{itemize}
